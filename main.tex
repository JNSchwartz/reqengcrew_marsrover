\documentclass[12pt]{report}
\usepackage{listings}
\usepackage{underscore}
\usepackage{vhistory}
\usepackage{setspace}
\usepackage{listings}
\usepackage{xparse}
\usepackage{xcolor}

\definecolorset{rgb/hsb/cmyk/gray}{}{}%
 {red,1,0,0/0,1,1/0,1,1,0/.3;%
  green,0,1,0/.33333,1,1/1,0,1,0/.59;%
  blue,0,0,1/.66667,1,1/1,1,0,0/.11;%
  brown,.75,.5,.25/.083333,.66667,.75/0,.25,.5,.25/.5475;%
  lime,.75,1,0/.20833,1,1/.25,0,1,0/.815;%
  orange,1,.5,0/.083333,1,1/0,.5,1,0/.595;%
  pink,1,.75,.75/0,.25,1/0,.25,.25,0/.825;%
  purple,.75,0,.25/.94444,1,.75/0,.75,.5,.25/.2525;%
  teal,0,.5,.5/.5,1,.5/.5,0,0,.5/.35;%
  violet,.5,0,.5/.83333,1,.5/0,.5,0,.5/.205}%
\definecolorset{cmyk/rgb/hsb/gray}{}{}%
 {cyan,1,0,0,0/0,1,1/.5,1,1/.7;%
  magenta,0,1,0,0/1,0,1/.83333,1,1/.41;%
  yellow,0,0,1,0/1,1,0/.16667,1,1/.89;%
  olive,0,0,1,.5/.5,.5,0/.16667,1,.5/.39}
\definecolorset{gray/rgb/hsb/cmyk}{}{}%
 {black,0/0,0,0/0,0,0/0,0,0,1;%
  darkgray,.25/.25,.25,.25/0,0,.25/0,0,0,.75;%
  gray,.5/.5,.5,.5/0,0,.5/0,0,0,.5;%
  lightgray,.75/.75,.75,.75/0,0,.75/0,0,0,.25;%
  white,1/1,1,1/0,0,1/0,0,0,0}



\usepackage[bookmarks=true]{hyperref}
\hypersetup{
    bookmarks=false,    % show bookmarks bar?
    pdftitle={Software Requirement Specification},    % title
    pdfauthor={Yiannis Lazarides},                     % author
    pdfsubject={TeX and LaTeX},                        % subject of the document
    pdfkeywords={TeX, LaTeX, graphics, images}, % list of keywords
    colorlinks=true,       % false: boxed links; true: colored links
    linkcolor=blue,       % color of internal links
    citecolor=black,       % color of links to bibliography
    filecolor=black,        % color of file links
    urlcolor=purple,        % color of external links
    linktoc=page            % only page is linked
}%

\usepackage{geometry}
\geometry{margin=1.5in}
\onehalfspacing


\usepackage{atbegshi}% http://ctan.org/pkg/atbegshi
\AtBeginDocument{\AtBeginShipoutNext{\AtBeginShipoutDiscard}}


\title{%

 


\Huge{Követelményelemzés}\\
\vspace{2cm}

Marsjáró jármű\\
\vspace{2cm}

\LARGE{Verzió 1.0}
\vspace{2cm}

\small{
Készítették:\\
Cseresnyés Kristóf: @aPisC\\
Fikó Róbert: @robertfiko\\
Gőgös Márton: @Marzyh\\
Hosek Henrietta: @hosekhenrietta\\
Kurucz Ádám: @engreyight\\
Lippa Katalin: @katilippa\\
Miskolczi Péter: @BreadKingGod\\
Monhor Hubert: @dopamean\\
Novák-Schwartz József: @JNSchwartz\\
Szabó Tihamér: @SzaboTihi14\\
Szerednik László Tamás: @szeredniklaszlo\\
}
}
\date{}
\usepackage{hyperref}



\begin{document}

\newcommand{\reqid}[1]{\colorbox{lightgray}{#1}}
\maketitle

\tableofcontents
\begin{versionhistory}
  \vhEntry{1.0}{22.10.23}{Novák-Schwartz József}{1st revision}
  \vhEntry{1.1}{22.11.06}{Robert Fiko}{Subsystems, risk analysis and formatting}
  \vhEntry{1.2}{22.11.09}{BreadKingGod}{Meghajtas}
  \vhEntry{1.3}{22.11.11}{szerendiklaszlo}{Navigation, sensors}
  \vhEntry{1.4}{22.11.12}{engreyight}{Drill}
  \vhEntry{1.5}{22.11.12}{szabotihi}{Supply, food, drink}
  \vhEntry{1.6}{22.11.12}{Novák-Schwartz József}{Spacesuit}
  \vhEntry{1.6.1}{22.11.13}{Robert Fiko}{Merge and review}
  \vhEntry{1.6.2}{22.11.13}{engreyight}{Glossary}
  \vhEntry{1.7}{22.11.13}{apisc}{Strength requirements}
  \vhEntry{1.8}{22.11.13}{katilippa}{Repairablity and upgradability}
  \vhEntry{1.9}{22.11.20}{BreadKingGod}{Meghajtas 2.0}
  \vhEntry{1.9.1}{22.11.13}{Robert Fiko}{Merge and review}



  

\end{versionhistory}
\chapter{Projekt bemutatása}
A NASA a Mars emberekkel történő felfedezéséhez magáncégek pályázatát várja a szükséges hardver és szoftverelemek kifejlesztéséhez az alábbi témakörökben:
\begin{enumerate}
  \item Marsbázis
  \item \textbf{Marsjáró jármű}
  \item Ellátmányűrhajó
  \item Mars-járó jármű
\end{enumerate}

Ebben a projektben 2 Mars-járó járműre készítünk specifikációt az alábbiakat figyelembe véve:
\begin{itemize}
  \item A járműveknek maximum 4 ember ellátását kell biztosítaniuk a Mars-járó missziók
időtartamára
  \item A jármű élettartama minimum 5 év legyen a Marsi körülmények között
  \item A járműnek lehetőséget kell biztosítania az űrhajósok kiszállására és beszállására
közvetlenül a Marsra vagy a marsi bázisra
  \item A jármű képes legyen a misszióhoz szükséges segédfelszerelések szállítására, vizsgálatok
elvégzésére, valamint a begyűjtött kőzetminták szállítására
  \item Az eszközhöz csatlakozzon egy repülni képes, felderítő drón
  \item A járműnek képesnek kell lennie kommunikálnia a Marsbázissal és a Mars körül keringő
átjátszó műholdakkal valamint űrállomással
  \item Végezzen kockázatanalízist és a szükséges alrendszereknél alkalmazzon redundanciát
  \item A főbb alrendszereknek javíthatónak és cserélhetőnek kell lennie az ellátmány bázis
alkatrészei alapján
\end{itemize}

\chapter{Követelmények}
\section{Követelmények számozása és rendszere}
A követelmények egyedi azonosítására az alábbi formátumot használjuk \\ 
\textbf{A_ID}, ahol \textbf{A} annak a részterületnek az azonosítója, ahova a követelmény tartozik, \textbf{ID} pedig az azon belüli követelmény azonosító.

\section{Megvalósítás ütemterve}

A megvalósítás ütemtervet, a marsbázis csapattal egyeztetve, az alábbiak szerint állítottuk fel.

\begin{itemize}
  \item Tervezési fázis: 2025. május 10
  \begin{itemize}
    \item A jármű részletes terveinek el kell készülnie.
  \end{itemize}
  \item Első prototípus: 2028. május 10
  \begin{itemize}
    \item Alapvető motorikus funkciók
    \item Dokkolás, légzsilip
    \item kommunikáció
    \item robot kar
  \end{itemize}
  \item Második prototípus: 2030. május 10
  \begin{itemize}
    \item Kisebb hibák még elfogadhatóak
    \item A véglesges specifikációt be kell tartalékmoduljainak
    \item Legalább 95\%-ban legyen megvalósítva
  \end{itemize}
  \item Tesztelés a Földön: 2031. május 10
  \begin{itemize}
    \item sarkvidéki, sivatagi és vulaknikus körülmények között
  \end{itemize}
  \item Marsi leszállítás: 2035. május 10
\end{itemize}


\section{Mérnöki követelmények}
\begin{itemize}
  \item A marsjáró mérete x*y*z. TODO:
  \item A marsjáró kereke (?). TODO:
  \item A marsi körülményeknek ellenálló anyag ?titánium?, ebböl készül a külsö ?n? réteg. TODO:
\end{itemize}

\subsection{A jármű alrendszerei}\label{subsystems}

A biztonságos és megbízható működés, illetve a modularitás elveinek betartása végett a 
járművet alrendszerekre bontjuk.

\subsubsection{Navigáció és szenzorok}
A navigációs alrendszer célja a vezetőt ellátni információkkal a terepviszonyokról, útvonalat tervezni a marsi felszínen, illetve egy bizonyos szintű önvezetést megvalósítani.

\noindent\textbf{Szükséges moduláris érzékelők és eszközök}
\begin{itemize}
  \item \reqid{NS_1} műholdas antennák
  \item \reqid{NS_2} giroszkóp és gyorsulásmérő
  \item \reqid{NS_3} motorhő-érzékelő
  \item \reqid{NS_4} több irányba tekintő ultrahangos szenzorok és közelség- és mélységérzékelők
  \item \reqid{NS_5} 360 fokos és 3D térlátást biztosító nagy felbontású kamerák 
  \item \reqid{NS_6} számítógépes egység (az adatok feldolgozásához)
\end{itemize}

\noindent\textbf{Biztosított funkciók}
\begin{itemize}
  \item \reqid{NS_7} helyzetmeghatározás: antennák által az űrbázis marsi műholdjaira csatlakozva
  \item \reqid{NS_8} térkép felépítése: műholddal készült felvételekkel, marsjáró által már bejárt helyekről készült adatrögzítés által
  \item  \reqid{NS_9} környezetfelismerés: akadálymentes meghajtás szenzorok és kamerák által \begin {itemize}
      \item lehetséges-e a kívánt dőlésszögben való haladás
      \item megléphetetlen akadályok detektálása
    \end{itemize}
\end{itemize}

\noindent\textbf{Különleges irányelvek}
\begin{itemize}
  \item alacsony üzemanyagszint: energiahasználat csökkentése a nem szükséges szenzorok kikapcsolásával, ezáltal növelni az esélyét a bázisra visszatérésnek
  \item útvonalon be nem tervezett akadály: újratervezés biztonságos útvonalat keresve
\end{itemize}

Becslések, gyártás és beszerzés, valamint ezek ledelegálása:
\begin{itemize}
  \item antennák, kamerák és szenzorok beszerzése: \begin{itemize}
      \item megfelelő mennyiségű éles- és tartalékkészlet megrendelése
      \item kapcsolatfelvétel az elvárt minőséget biztosító, és minél kedvezőbb árat kiszabó beszállítókkal
      \item időbecslés: 1.5 hónap
      \item pénzbecslés: 350 AK
    \end{itemize}
  \item \reqid{NS_10} navigációs szoftver: \begin{itemize}
      \item előfeltétel: navigációs szenzorok beszerzése, programot elkészítő csapat megbízása
      \item időigény: 6 hónap
      \item pénzbecslés: 10000 AK
    \end{itemize}
  \item összesített becslés: \begin{itemize}
      \item előfeltétel: navigációs szenzorok beszerzése, programot elkészítő csapat megbízása
      \item időigény: 7.5 hónap
      \item pénzbecslés: 10350 AK
    \end{itemize}
\end{itemize}

Ahol 1 Arany Krajcár (AK) = 1 EUR

\subsubsection{A fúró és szenzorai, vizsgálatok}

A marsjárónak több különböző fúrófejjel kell rendelkeznie, melyek cserélhetők. Ezeket használja egyrészt 30 cm mély lyukak fúrására (\reqid{F_1}), illetve a felszín lekaparására. (\reqid{F_2}) Az így keletkezett lyukak alján ezután párologtatást (\reqid{F_3}) végez, és az így kiváló anyagokat összegyűjti. (\reqid{F_4}) Emellett képes a felszínen már jelenlévő és a kaparással előállított regolitot felszívni, illetve esetlegesen maximum 50x25x25cm-es már szabadon lévő kőzetdarabokat begyűjteni. (\reqid{F_5})  A marsjáró összesen 800kg kőzetminta tárolására és szállítására képes. (\reqid{F_6}) 

A minták tárolására két fajta tároló használható. Egy egyszer használatos, melyet a minták felhasználása után ki kell dobni, vagy egy speciális, vegyszeres tisztítás után újrahasznosítható. Szükség van emelett olyan "tárolókra", amelyekbe nem kerülnek minták, csupán arról adnak tanubizonyságot, hogy a kinyerés során nem szennyeződött a többi minta. (\reqid{F_7}) 

A fúró egy robotkarra van rögzítve, mely számos szabadsági fokkal, vagy izülettel rendelkezik. 

Első a váll, amely minden irányban mozgatható egy megadott tartományban. (\reqid{R_1}) Ezt követi a könyök, amely hajlítható és egy 360 fokban elforgatható csukló. (\reqid{R_2}) Ennek a végén található egy fogó, melynek egyik ága a fúró, a másik pedig egy kamera, amely a fúrást segíti.  Ehhez egy lámpa is tartozik, és normál fénytartományban működik. (\reqid{R_3}) Emellett található rajta egy biztonsági kontaktszenzor, amely véd a berendezés meghibásodása ellen. (\reqid{R_4})

A marsjárón a fúrófejhez tartozó kamera mellett található még 4, ebből 2 előre, 2 pedig hátrafelé néz. Ezek kellő távolságban vannak egymástól ahhoz, hogy a mélységérzékelés működjön. Minden kamerához tartozik 1-1 lámpa, és ezek az infravörös tartományban is működnek. (\reqid{NS_11})

Az összegyűjtött mintákat el lehet tárolni, és a bázisra szállítani, vagy akár már a marsjáróban fel lehet őket nyitni és alapvető vizsgálatokat végezni rajtuk.

A vizsgálatok fő célja a bolygó geológiájának és történelmének feltárása, múltbéli élhetőségének vizsgálata, és víz és sejtes élet nyomainak keresése.

\subsubsection{Egyéb szenzorok}
A marsjáró optimális külső és belső állapotainak ellenőrzéséhez szükség van a navigációtól független szenzorokra is.

Ezekhez szükségesek:
\begin{itemize}
  \item külső szenzorok: \begin{itemize}
      \item \reqid{S_1} külső hőmérő
      \item \reqid{S_2} napállást vizsgáló szenzor
      \item \reqid{S_3} légnyomásmérő
    \end{itemize}
  \item belső szenzorok: \begin{itemize}
      \item \reqid{S_4} marsjáró és szkafanderek töltöttség-mérője
      \item \reqid{S_5} oxigénszint-mérő
      \item \reqid{S_6} nyomásszint-mérő
      \item \reqid{S_7} belső hőmérő (ezáltal ellenőrizve a megrendelő által elvárt belső hőt)
      \item \reqid{S_8} kőzetminta mérleg (hogy ne lépjük túl a megrendelő által maximálisan elvártan szállított kőzetek tömegét)
    \end{itemize}
\end{itemize}

Ezek által megállapítható, hogy:
\begin{itemize}
  \item alkalmasak-e a külső körülmények a marsjáró épségben maradását tekintve
  \item alkalmasak-e a belső körülmények marsjáróban lévő emberek számára
\end{itemize}

\subsubsection{Meghajtás}

Komponensek:

\begin{itemize}
    \item \reqid{M_1} Kerék:
        \begin{itemize}
            \item Levegőmentes kerék
            \item Alumíniumból kell készíteni
            \item A tapadás érdekében kapcsokkal kell rendelkeznie
            \item A rugós tapadásért titánium külőket kell bele építeni
        \end{itemize}
    \item \reqid{M_2} Futómű:
        \begin{itemize}
            \item Fémből kell készíteni
            \item Teherbírásának akkorának kell lenni, hogy elbírja a kocsi tömegét
        \end{itemize}
    \item \reqid{M_3} Motor:
        \begin{itemize}
            \item Hidrogénmotor
            \item Bázistól feltöltéskor vizet és energiát kapunk, a vizet a kapott energia segítségével kell oxigénre és hidrogénre bontani
            \item A motor belsejében a hidrogént az oxigén segítségével elégetjük, a felszabaduló energiát használja a motor meghajtásra, a melléktermék vizet meg emberi fogyasztásra alkalmazható
        \end{itemize}
    \item \reqid{M_4} Tank:
        \begin{itemize}
            \item Méretének akkorának kell, hogy 200 km üzemanyag elférjen benne
            \item Két azonos méretű üzemanyag cellából álljon
            \item A két üzemanyag cellából párhuzamosan fogyasszon üzemanyagot
            \item Az üzemanyag cellák legyenek képesek önállóan is ellátni az eszközt
            \item Feltölthető legyen a bázisról
            \item Legyen egy rendszer ami méri a tank az üzemanyagcellák töltöttségét
            \item Ez a rendszer értesítse a vezetőt, amikor a tank töltöttség félig van
            \item Az üzemanyagcellák dokkolva tölthetőnek kell lenniük
        \end{itemize}
    \item \reqid{M_5} Közlekedés:
        \begin{itemize}
            \item El kell érnie a  25 km/h sebességet
            \item Képesnek kell lennie 20 nap alatt 20 km-t megtenni
            \item A bevehető meredekségének 20-30-osnak kell lennie
            \item A járműnek képesnek kell lennie oldalazva közlekedni
        \end{itemize}
\end{itemize}

Becslések és gyártás:

\begin{itemize}
    \item Kerék:
        \begin{itemize}
            \item Beszerez:
                \begin{itemize}
                    \item Titánium küllők: Total Titan Titanium
                    \item Aluminum: Cosmo Aluminum
                \end{itemize}
            \item Gyártás: mi gyártjuk
            \item Idő becslés: 2 hónap
            \item Pénz becslés: 450 AK
        \end{itemize}
    \item Tank:
        \begin{itemize}
            \item Gyártás: delegáljuk a HCella cégnek
            \item Idő becslés: 1 hónap
            \item Pénz becslés: 100 AK
        \end{itemize}
    \item Motor:
        \begin{itemize}
            \item Gyártás: delegáljuk a Massive Motor cégnek
            \item Előfeltétel: 
                \begin{itemize}
                    \item Tank
                \end{itemize}
            \item Idő becslés: 1 év
            \item Pénz becslés: 10000 AK
        \end{itemize}
    \item Futómű:
        \begin{itemize}
            \item Beszerez:
                \begin{itemize}
                    \item Aluminum: Cosmo Aluminum
                \end{itemize}
            \item Gyártás: mi gyártjuk
            \item Előfeltétel: 
                \begin{itemize}
                    \item Motor
                    \item Kerék
                \end{itemize}
            \item Idő becslés: 6 hónap
            \item Pénz becslés: 5500 AK
        \end{itemize}
    \item Összesített:
        \begin{itemize}
            \item Idő becslés: 1 év, 9 hónap
            \item Pénz becslés: 16050 AK
        \end{itemize}
\end{itemize}

\subsubsection{Létfentartás}

TODO: Req Id-k

Természetesen az emberi személyzet végett, a létfenntartó rendszerek elengedhetetlenek. XYZ típusú levegő szűrő és széndioxid kivonó rendszerrel és XYZ típusú vízújrahasznosító rendszerrel fogjuk felszerelni. Az utóbbi képes a levegő páratartalmát kivonni, illetve vizeletet tisztítani, így végül fogyasztásra alkalmas ivóvizet kapunk.

Az asztronautáknak a megfelelő hőmérséklet biztosítására az XYZ típusú fűtő berendezést szereljük, mivel a marsi hőmörséklet nem haladja meg a 21 Celsius fokot, így fűtés biztosítása elegendő.

Forrás: https://www.weather.gov/fsd/mars
TODO: cite

\subsubsection{Telekommunikáció}

TODO: Req Id-k

A telekommunkációhoz XYZ típusú, PRZ paraméterű antennákat és XYZ típusú jelfeldolgozó egységeket fogunk használni.

TODO: kérdés, bázissal kommunikációs frekvencia

\subsubsection{Szkafander/űrruha és a mars járó kapcsolata}
\begin{itemize}

  \item \reqid{Z_1} A bázissal megegyezö módon töltjük a szkafander elektromos akkumulátorát és oxigénpalackját is a roverben.
  \item \reqid{Z_2} A jármüvön egy fajta sürített \(O_2\) palack, amiből a járón belül is keverjük a levegöt, és a palackot is tudjuk tölteni, ezt pedig a bázison töltjük újra. Biztonsági redundancia szempontból persze lehet több palack, de praktikus, ha csak egyfajta, és mindkét funkciót képes biztosítani.
  \item \reqid{Z_3} Töltöaljzat és oxigéncsap biztosítása.
  \item \reqid{Z_4} Szkafander töltö alkatrész, oxigén palack szükítö alkatrész.
  \item Ezek beszerelése a teljes elektromos és oxigén kivitelezési körök után kezdödhet, és átadásig elég befejezödnie.
  \item \reqid{Z_5} A tiszta oxigén légzése elött egy ú.n. elölégzésre van szükség, ami a nitrogént kiüríti a szervezetböl, így megelözve a sérüléseket. Ehhez szükséges vákuumkamra biztosítása. Javaslom a mars felé nyíló zsiliprendszer vákuumképzéssel felruházását.
  \item \reqid{Z_6} Oxigén nyomás kritikus csökkenés esetén, vagy kritikus áramcsökkenés esetén azonnali visszatérés megkövetelése. Fel nem töltött szkafanderek esetén töltés azonnali megkövetelése, vagy kötelezö visszatérés. A humán biztonság megköveteli, hogy mindig legyen töltött szkafander evakuációhoz.
\end{itemize}

\subsubsection{Energiaellátás}

TODO:

ötletek:
- power
  - managing the power in the rover
  - how electric power is flowing in the vechile

\subsubsection{Drón}

A marsjáróhoz tartozik egy drón. Ez egy a földön hétköznapi emberek által is használt drónhoz hasonló működési elvű eszköz. A feladata felderítés. A marsjáróhoz kapcsolódik és egy-egy misszió során gyűjt képanyagokat a környezetről.

A drón repülésre képes, ehhez két koaxiális propellerrel van felszerelve. (\reqid{D_1}) Ezek használatosak helikoptereknél is, ugyanis megkönnyítik a manőverezést. Mind a hajtásért felelős szerkezet, mind maga a drón váza és egyéb építőegységei - hasonlóan a marsjáróhoz - olyan anyagokból kell készüljenek, amik kibírnak bármilyen misszió során tapasztalt szélsőséges marsi időjárási viszonyokat. (\reqid{D_2})\\

A drón célja, hogy felvételeket készítsen a környezetről. Ehhez egy általános kamerával van felszerelve, amivel normál látható színtartományban képes felvételeket csinálni. (\reqid{D_3}) Ezen kívül szükséges felszerelése még egy infrakamera is. (\reqid{D_4}) A felvételeket elsődlegesen a drón továbbítja a marsjáróra, hogy azok azonnal feldolgozhatóak, vizsgálhatóak legyenek. A képanyagok tárolására rendelkezik egy saját átmeneti adattárolóval is, ami a FIFO módszeren alapul. (\reqid{D_5})\\

Működését tekintve elektromos hajtású. (\reqid{D_6}) Irányítása történhet a marsjáróról manuálisan (\reqid{D_7}), azonban elsősorban önműködő. (\reqid{D_8}) Képes kell legyen magától eltalálni a felderíteni kívánt helyszínre, irányba, valamint onnan automatikusan visszajutni a marsjáróra.(\reqid{D_9}) A hatótávolsága 4500 méter. (\reqid{D_10})ALapvető biztonsági fukciókkal kell rendelkezzen, mint a levegőben való egyhelyben repülés (várakozás), egyensúlyozás. (\reqid{D_11})\\

Tárolása a marsjáró külsején történik egy akkumulátorral felszerelt dokkolón. Töltése a járműről történik a dokkolón keresztül. (\reqid{D_12})

\section{Humán követelmények}
\begin{itemize}
  \item \reqid{H_1} A marsjárón szkafander nélküli környezetet kell biztosítani. (Szobahömérséklet, levegö.)
  \item \reqid{H_2} A marsjáron biztosítani kell a szkafanderek töltését, a bázissal egyezö módon.
  \item \reqid{H_3} A küldetésekhez elegendö ételek tárolása és ivóvíz biztosítása
  Ezt a problémát két részre bontjuk, az első maga a tároló egység, amelyben az étel, és a szükséges ivóvíz van. \reqid{H_4} A másik pedig a vizelet disztillálás, mellyel újrahasznosíthatjuk a vizeletet, izzadtságot. 
  \begin{itemize}
    \item Vizelet disztilláló \reqid{H_5} \\ 
    A NASA folyamatosan fejleszti azt az eszközt, amely a nemzetközi űrállomáson is megtalálható. A mai napig igyekeznek a vízszállítást a lehető legjobban csökkenteni. Egy ilyen disztillálóra lenne szükségünk tőlük, mely a legénység vízszükségleteinek 60\%-át tudja ellátni.
    Egy ilyennek a mérete 35-55 kilogram, térfogatban pedig 1 méter széles, 1,7 méter magas és fél méter a mélysége.
    Ennek elhelyezését a marsjáróban a egy toalett méretű helységben helyeznénk el. Ez a helység egy külön modul.
    Ez a gépezet naponta 9 kg vizeletet képes feldolgozni, ez kevesebb, mint amit 4 ember 16 óra alatt termel. A maradék vizeletüket el kell tárolni, hogy a későbbiekben felhasználhassák, akár a bázison, akár egy másik küldetésen. A vizelettárolást egy steril tartályban kell tárolni, amelynek hőmérséklete nem érheti el a 40 Celsius-fokot. Ennek megoldására, az újrahasznosítóval egy teremben kellene létrehozni egy toalett szerűséget, amely az említett tartályba gyűjti a vizeletet. Ezen tartálynak 380 literesnek kell lennie, hogy minden vizeletet megőrizzen, viszont hatalmas méretei miatt ezt 240 literesre tervezzük. A többi vizelet elveszhet. Ennek beszerzéséhez 400-500 euróra van szükség.
    \item Tároló egység \reqid{H_6} \\ 
    A tároló raktár egy külön helység lesz a marsjárón, melyben különleges polcok vannak, amelyeknek az ételt tartalmazó részein a hőmérséklet állítható. Emelett olyan módon tartalmazza az ételt, hogy az a marsjáró különböző szögű mozgásaikor se essen le a polcról. Maga a moduláris terem beszerzése/beszerelése 4500-5000 euró. Ezen terem méretei: 2m belmagasság, 3 méter széles és 2 méter mély. A polcoknak együtt 140 kilogramm étel eltárolását kell biztosítani. Ezen polcok a teremben alul és felül is csatlakozzanak a felülettel. Egy ilyen polcnak az előállítása és beszerelésének költsége 350-400 euró, mivel négy ilyen polc van, ezért 1600 euró. \\
    A teremben szükség van még egy nyílásra, amely a 80 literes ivóvíz tartályhoz csatlakozik, ez a víz 40\% annak amelyet 4 ember megiszik 20 nap alatt. A többi víz a vizelet újrahasznosítással állítódik elő. A vizes tartály beszerzése és beszerelése eggyüttesen 3000-3500 euróba kerül.  
  \end{itemize}  
  \item Alapvetö higiénés és rekreációs szükségletek biztosítása feltétel. \reqid{H_7}
\end{itemize}
\section{Kutatási követelmények}
\begin{itemize}
  \item \reqid{K_1} A marsjáróra mintavételezéshez szükséges egy fúrót felszerelni.
  \item \reqid{K_2} A begyüjött mintaásványokon kisebb laborvizsgálatokat el kell tudni végezni, mint pl. centrifugálás.
  \item \reqid{K_3} Z TODO: köbméter ásvány a ázisra való visszaszállítását kell biztosítani.
\end{itemize}


\section{Javíthatóság és fejleszthetőség}

A jármű felépítését modulárisra kell tervezni, biztosítva, hogy meghibásodás esetén bármely alkatrész könnyen hozzáférhető és cserélhető legyen. (\reqid{J_1}) Ezen felül a vizsgálatokhoz szükséges eszközök újabb típusra történő leváltására is lehetőség kell legyen. (\reqid{J_2}) \\
A létfenntartást, helyzetmeghatározást és kommunikációt biztosító rendszerek biztonsági okokból a missziók alatt is javíthatók kell legyenek, (\reqid{J_3}) a marsjárón az ehhez szükséges pótalkatrészek és eszközök rendelkezésre kell álljanak. (\reqid{J_4}) \\
A pótkerekek kivételével a nagyobb méretű és nem létfontosságú rendszerek tartalékmoduljainak elegendő a bázison elérhetőnek lenni. (\reqid{J_5}) \\
Mivel bizonyos komponensek csak kívülről hozzáférhetők és cserélhetők, ezért a kötőelemek és a hozzájuk tartozó szerszámok megtervezésénél elengedhetetlen szempont, hogy az asztronauták mozgása korlátozva és nehezítve lesz a szkafander által. Mivel a bázison nem lesz garázs, minden külső javításnak elvégezhetőnek kell lennie űrruhában. (\reqid{J_6})

\section{Strapabírósági követelmények}
A marsjárónak képesnek kell lennie a zord marsi körülményeknek ellenállni hosszú időn keresztül, mivel a marsi időjárási viszonyok különböző jelentős nehézségeket tartogatnak. Egyik ilyen nehézség amire fel kell készülni, a különböző sugárzások emelkedett szintje. (\reqid{T_1}) A Mars nem rendelkezik olyan légkörrel és 
mágneses mezővel, mint a Föld, ezért fokozottan ügyelni kell a marsjáró megtervezésénél, hogy a sugárzás ne csökkenthesse az eszköz tervezett élettartamát. 

A hőmérséklet tekintetében a járműnek fel kell készülnie a Marson uralkodó kifejezetten alacsony hőmérséklet elviselésére is. A felszíni hőmérséklet körülbelül  -120 és 20 °C között mozoghat, így fel kell készülni az extrém hideg okozta problémákra. (\reqid{T_2}) A hőmérséklet ilyen mértékű ingadozásánál figyelni kell a folyamatos hőtágulás okozta kopás és elhasználódás minimalizálására is. (\reqid{T_3}) Továbbá a kopáshoz hozzá járulhatnak a különböző rengések és homokviharok, amik a bolygó felszínén uralkodnak.

Összességében az eszköz épségének és használhatóságának megőrzésére fel kell szerelni egy olyan védelemmel, ami egyszerre tud ellenállni szélsőséges hőmérsékleti viszonyoknak és különböző mechasnikai behatásoknak is.


\section{Biztonsági követelmények}

A biztonságos misszióhoz elengedhetetlen a jármű kockázat analaízise, mely során annak alrendszereit feltérképezzük és a kritikusakat meghatározzuk. 

\subsection{Kockázat analízis}



A kockázat analízis célja, hogy a \ref{subsystems} fejezetben részletezett alrendszerek közül melyek a kritkus rendszerek és mennyire fontosak. A legnyílvánvalóbb az energia ellátó alrendszer, hiszen ha nincsen enerigai hiába van létfentartó rendszer vagy bármi más. Ha nincsen energia, akkor nem várható el egyik alrendszertől, vagy tartalékrendszertől sem, hogy működjön.

TODO: honna van energia, milyen energia hordozó, aztán tartalék akkumlátor, esetleg dízel generátor

Miután a járműnek van redundás energiaforrása, a létfenntartó rendszert fontos megvizsgálni. Az tényként kezelendő, hogy a fedélzeten utazó űrhajósokat életben kell tartni, így ezen rendszereknek is szükséges tartalékot képezni, legalább a levegő- és vízkezelés illetve a fűtés szintjén. 

\textit{Megjegyzés: A rover tartalékrendszerein túl az utasok bármikor felvehetik a szkafander sisakjukat, így akár egy harmadfokú renduncacnia is elérhető.}

A fent említett rendszereken túl, a harmadik a sorban a telekommunkációs rendszerek, hogy a kutatók tudjanak segítséget kérni a bázistól, vagy a másik rovertól amennyiben szükséges.

TODO: rendszerek megvalósítása







\begin{itemize}
  \item Az esetleges katasztrófaesemények esetén is biztosítani kell a személyzet ->bázisra való visszajutását. TODO:
\end{itemize}
\section{Utókövetelmények}
\begin{itemize}
  \item \reqid{U_1} A marsjáró leszerelése és kezelése mint hulladék.
  \item \reqid{U_2}A marsjáró által okozott esetleges természeti károk helyreállítási terve.
\end{itemize}

\chapter{Szójegyzék}
\section{Marsi fogalmak}
\begin{itemize}
  \item A marsi gravitáció x.
  \item A marsi év y. A fentiekben mi földi évben számolunk.
  \item A mars átlaghömérséklete z. k és j között ingadozik.
  \item A marsi UV l. Ez a földihez képest f/l, ami az eszközök élettartamát befolyásolja.
\end{itemize}

\section{További fogalmak}

\begin{itemize}

  \item \textbf{koaxiális propellerek}: Egy tengelyen egymás fölött elhelyezett két darab propeller, amik ellentétes irányba forognak. Ezzel nagyban javul a hajtás és manőverezés.
  \item \textbf{propeller}: Légcsavar. Repülőgépek, helikopterek és egyéb repülő eszközök (ezesetben drón) körében használt erőátviteli megoldás. A motor teljesítményét közvetíti a hordozó közegre (levegő).
  \item \textbf{FIFO módszer}: "First in firts out", azaz amelyik adat (képfelvétel) előszür került tárolásra azt továbbítja / az vehető ki először.

  \item \textbf{fogalom}: Lorem ipsom dolor sit amet TODO:
  \item \textbf{regolit}: Szilárd kérgű égitesteken a felszíni törmelékes, általában laza szerkezetű kőzetrétege. A talaj is a regolitnak nevezhető, de regolitról leginkább akkor beszélünk, ha a regolitréteg nem talajosodott.
  \item \textbf{geológia}: földtan
  \item \textbf{hőpárologtatás}: a kiásott lyuk mélyére egy hevítőrudat juttatnak, a hő hatására a kőzetben reakciók hajtódnak végre, anyagok válnak ki, amelyek vizsgálatával fontos információhoz juthatunk

\end{itemize}








\end{document}
